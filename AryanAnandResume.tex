\documentclass[10pt, letterpaper]{article}

% Packages:
\usepackage[
    ignoreheadfoot, % set margins without considering header and footer
    top=2 cm, % seperation between body and page edge from the top
    bottom=2 cm, % seperation between body and page edge from the bottom
    left=2 cm, % seperation between body and page edge from the left
    right=2 cm, % seperation between body and page edge from the right
    footskip=1.0 cm, % seperation between body and footer
    % showframe % for debugging 
]{geometry} % for adjusting page geometry
\usepackage{titlesec} % for customizing section titles
\usepackage{tabularx} % for making tables with fixed width columns
\usepackage{array} % tabularx requires this
\usepackage[dvipsnames]{xcolor} % for coloring text
\definecolor{primaryColor}{RGB}{0, 0, 0} % define primary color
\usepackage{enumitem} % for customizing lists
\usepackage{fontawesome5} % for using icons
\usepackage{amsmath} % for math
\usepackage[
    pdftitle={John Doe's CV},
    pdfauthor={John Doe},
    pdfcreator={LaTeX with RenderCV},
    colorlinks=true,
    urlcolor=primaryColor
]{hyperref} % for links, metadata and bookmarks
\usepackage[pscoord]{eso-pic} % for floating text on the page
\usepackage{calc} % for calculating lengths
\usepackage{bookmark} % for bookmarks
\usepackage{lastpage} % for getting the total number of pages
\usepackage{changepage} % for one column entries (adjustwidth environment)
\usepackage{paracol} % for two and three column entries
\usepackage{ifthen} % for conditional statements
\usepackage{needspace} % for avoiding page brake right after the section title
\usepackage{iftex} % check if engine is pdflatex, xetex or luatex

% Ensure that generate pdf is machine readable/ATS parsable:
\ifPDFTeX
    \input{glyphtounicode}
    \pdfgentounicode=1
    \usepackage[T1]{fontenc}
    \usepackage[utf8]{inputenc}
    \usepackage{lmodern}
\fi

\usepackage{charter}

% Some settings:
\raggedright
\AtBeginEnvironment{adjustwidth}{\partopsep0pt} % remove space before adjustwidth environment
\pagestyle{empty} % no header or footer
\setcounter{secnumdepth}{0} % no section numbering
\setlength{\parindent}{0pt} % no indentation
\setlength{\topskip}{0pt} % no top skip
\setlength{\columnsep}{0.15cm} % set column seperation
\pagenumbering{gobble} % no page numbering

\titleformat{\section}{\needspace{4\baselineskip}\bfseries\large}{}{0pt}{}[\vspace{1pt}\titlerule]

\titlespacing{\section}{
    % left space:
    -1pt
}{
    % top space:
    0.3 cm
}{
    % bottom space:
    0.2 cm
} % section title spacing

\renewcommand\labelitemi{$\vcenter{\hbox{\small$\bullet$}}$} % custom bullet points
\newenvironment{highlights}{
    \begin{itemize}[
        topsep=0.10 cm,
        parsep=0.10 cm,
        partopsep=0pt,
        itemsep=0pt,
        leftmargin=0 cm + 10pt
    ]
}{
    \end{itemize}
} % new environment for highlights


\newenvironment{highlightsforbulletentries}{
    \begin{itemize}[
        topsep=0.10 cm,
        parsep=0.10 cm,
        partopsep=0pt,
        itemsep=0pt,
        leftmargin=10pt
    ]
}{
    \end{itemize}
} % new environment for highlights for bullet entries

\newenvironment{onecolentry}{
    \begin{adjustwidth}{
        0 cm + 0.00001 cm
    }{
        0 cm + 0.00001 cm
    }
}{
    \end{adjustwidth}
} % new environment for one column entries

\newenvironment{twocolentry}[2][]{
    \onecolentry
    \def\secondColumn{#2}
    \setcolumnwidth{\fill, 4.5 cm}
    \begin{paracol}{2}
}{
    \switchcolumn \raggedleft \secondColumn
    \end{paracol}
    \endonecolentry
} % new environment for two column entries

\newenvironment{threecolentry}[3][]{
    \onecolentry
    \def\thirdColumn{#3}
    \setcolumnwidth{, \fill, 4.5 cm}
    \begin{paracol}{3}
    {\raggedright #2} \switchcolumn
}{
    \switchcolumn \raggedleft \thirdColumn
    \end{paracol}
    \endonecolentry
} % new environment for three column entries

\newenvironment{header}{
    \setlength{\topsep}{0pt}\par\kern\topsep\centering\linespread{1.5}
}{
    \par\kern\topsep
} % new environment for the header

\newcommand{\placelastupdatedtext}{% \placetextbox{<horizontal pos>}{<vertical pos>}{<stuff>}
  \AddToShipoutPictureFG*{% Add <stuff> to current page foreground
    \put(
        \LenToUnit{\paperwidth-2 cm-0 cm+0.05cm},
        \LenToUnit{\paperheight-1.0 cm}
    ){\vtop{{\null}\makebox[0pt][c]{
        \small\color{gray}\textit{Last updated in September 2024}\hspace{\widthof{Last updated in September 2024}}
    }}}%
  }%
}%

% save the original href command in a new command:
\let\hrefWithoutArrow\href

% new command for external links:
% ... [everything before \begin{document} remains unchanged]

\begin{document}
\newcommand{\AND}{\unskip
    \cleaders\copy\ANDbox\hskip\wd\ANDbox
    \ignorespaces
}
\newsavebox\ANDbox
\sbox\ANDbox{$|$}

\begin{header}
    \fontsize{25 pt}{25 pt}\selectfont Aryan Anand

    \vspace{5 pt}

    \normalsize
    \mbox{Bengaluru, India}%
    \kern 5.0 pt%
    \AND%
    \kern 5.0 pt%
    \mbox{\hrefWithoutArrow{mailto:aryananandxic07@gmail.com}{aryananandxic07@gmail.com}}%
    \kern 5.0 pt%
    \AND%
    \kern 5.0 pt%
    \mbox{\hrefWithoutArrow{tel:+91-720-403-61-77}{+91 7204036177}}%
    \kern 5.0 pt%
    \AND%
    \kern 5.0 pt%
    \mbox{\hrefWithoutArrow{https://aryan-codes.netlify.app/}{aryan-codes.netlify.app}}%
    \kern 5.0 pt%
    \AND%
    \kern 5.0 pt%
    \mbox{\hrefWithoutArrow{https://www.linkedin.com/in/aryan-anand-82b807279/}{linkedin.com/in/aryan-anand-82b807279/}}%
    \kern 5.0 pt%
    \AND%
    \kern 5.0 pt%
    \mbox{\hrefWithoutArrow{https://github.com/aryan-212}{github.com/aryan-212}}%
\end{header}

\vspace{5 pt - 0.3 cm}
\section{Education}

\begin{twocolentry}{
    Sept 2022 – May 2026
}
    \textbf{PES University}, B.Tech in Computer Science
\end{twocolentry}

\vspace{0.10 cm}
\begin{onecolentry}
    \begin{highlights}
        \item \textbf{Coursework:} Data Structures \& Algorithms, Computer Networks, Applied Cryptography, Big Data, Database Management, Database Technologies, Blockchain
        \hfill \textit{6.82/10}
    \end{highlights}
\end{onecolentry}

\vspace{0.30 cm}
\begin{twocolentry}{
    Mar 2020 – Mar 2022
}
    \textbf{Rajendra Vidyalaya}, X - XII
\end{twocolentry}
Physics, Chemistry, Mathematics, Computer Science
\hfill \textit{84.7\%}

\section{Experience}

\textbf{AI Systems Engineering Intern} \hfill \textit{Cosdata (Next-gen Vector Database) · Remote} \\
\textit{Jun 2025 – Present}
\vspace{-0.5em}
\begin{itemize}
  \setlength\itemsep{0.3em}
  \item Contributing to Cosdata’s \textbf{low-latency AI data platform}, with focus on ingestion, indexing, and search infrastructure.
  \item Built and optimized \textbf{streaming upsert APIs} and \textbf{hybrid search endpoints} using Rust + Actix.
  \item Worked on performance-critical paths involving \textbf{QPS benchmarking}, \textbf{BEIR dataset evaluation}, and \textbf{Recall@10 / nDCG} testing.
  \item Implemented fixes for index invalidation, duplicate vector insertions, and route hygiene.
  \item Collaborated with core engineers on profiling async tasks, reducing memory allocations, and improving concurrency behavior.
\end{itemize}

\section{Projects}

\begin{twocolentry}{
    \href{https://github.com/aryan-212/BashScripts}{aryan-212/BashScripts}
}
    \textbf{BashScripts - Collection of Custom Utility Scripts}
\end{twocolentry}

\begin{onecolentry}
    \begin{highlights}
        \item \textbf{BigData-Docker}: Simplified the Big Data elective setup via Docker-based environment.
        \item \textbf{PESULogin}: Automated captive portal login for college Wi-Fi using `curl`.
        \item \textbf{ArchHibernator}: Enabled seamless hibernation on Arch via swap/kern param detection.
        \item Tools Used: Bash, Docker, cURL, Linux System Utilities
    \end{highlights}
\end{onecolentry}

\vspace{0.2 cm}
\begin{twocolentry}{
    \href{https://github.com/aryan-212/DeepStalk.git}{aryan-212/DeepStalk}
}
    \textbf{DeepStalk - Multi-Threaded Dark Web Scraper}
\end{twocolentry}

\begin{onecolentry}
    \begin{highlights}
            \item Developed the world’s fastest open-source .onion crawler, leveraging \textbf{Rust’s async runtime}, Tokio for \textbf{concurrent networking}, and a high-performance frontier management system with \textbf{Bloom filters and priority queues}.
        \item Tools Used: Rust, Reqwest, Docker, ElasticSearch
    \end{highlights}
\end{onecolentry}

\vspace{0.2 cm}
\begin{twocolentry}{
    \href{https://github.com/aryan-212/BackendBro.git}{aryan-212/BackendBro}
}
    \textbf{BackendBro - AI-Powered Rust API Generator}
\end{twocolentry}

\begin{onecolentry}
    \begin{highlights}
        \item Created a \textbf{multi-agent AI system} that converts \textbf{natural language prompts} into fully functional \textbf{Rust APIs} with complete \textbf{CRUD operations}, automated \textbf{testing}, and \textbf{self-healing capabilities} for robust backend generation.

        \item Tools Used: Rust, Actix-web, Tokio, Serde, Reqwest
    \end{highlights}
\end{onecolentry}

\vspace{0.2 cm}
\begin{twocolentry}{
    \href{https://github.com/aryan-212/EmoStream}{aryan-212/EmoStream}
}
    \textbf{EmoStream - Real-time Emoji Processing System}
\end{twocolentry}

\begin{onecolentry}
    \begin{highlights}
        \item Built a high-performance, \textbf{event-driven system} using \textbf{Kafka} and \textbf{Flask} for real-time emoji broadcasting. Implemented \textbf{concurrent processing} with multiple subscribers, RESTful APIs for client management, and containerized deployment using \textbf{Docker}
        \item Tools Used: Python, Kafka, Flask, SparkSQL
    \end{highlights}
\end{onecolentry}

\section{Technologies}

\begin{onecolentry}
    \textbf{Languages:} Rust, Go, Python, SQL
\end{onecolentry}

\begin{onecolentry}
    \textbf{Technologies \& Tools:} Docker, Kubernetes, Git, GitHub Actions, Linux, Neovim, VS Code
\end{onecolentry}

\begin{onecolentry}
    \textbf{Backend:} Axum, Actix, PostgreSQL, MongoDB, Redis, Firebase, REST, WebSockets
\end{onecolentry}

\begin{onecolentry}
    \textbf{Distributed Systems:} Kafka, Hadoop, Spark, Flink, Redis Queue, Elasticsearch
\end{onecolentry}



\end{document}

